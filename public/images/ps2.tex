%me=0 student solutions, me=1 - my solutions, me=2 - assignment
\def\me{2}
\def\num{2}  %homework number
\def\due{Sept 18, 2024}  %due date
\def\course{CS 3950} %course name
\def\name{Student Name}
%
\iffalse

\fi
%
\documentclass[11pt]{article}
\usepackage{amsfonts}
\usepackage{latexsym}
\usepackage{tikz}
\usepackage{graphicx}
\usepackage{hyperref}


\newcommand{\handout}[5]{
   \renewcommand{\thepage}{#1, Page \arabic{page}}
   \noindent
   \begin{center}
   \framebox{
      \vbox{
    \hbox to 5.78in { {\bf \course} \hfill #2 }
       \vspace{4mm}
       \hbox to 5.78in { {\Large \hfill #5  \hfill} }
       \vspace{2mm}
       \hbox to 5.78in { {\it #3 \hfill #4} }
      }
   }
   \end{center}
   \vspace*{4mm}
}

\newcounter{pppp}
\newcommand{\prob}{\arabic{pppp}}  %problem number
\newcommand{\increase}{\addtocounter{pppp}{1}}  %problem number

\newcommand{\newproblem}[2]{\increase
\section*{Problem \prob~(#1) \hfill {#2}}}


\def\squarebox#1{\hbox to #1{\hfill\vbox to #1{\vfill}}}
\def\qed{\hspace*{\fill}
        \vbox{\hrule\hbox{\vrule\squarebox{.667em}\vrule}\hrule}}
\newenvironment{solution}{\begin{trivlist}\item[]{\bf Solution:}}
                      {\qed \end{trivlist}}
\newenvironment{solsketch}{\begin{trivlist}\item[]{\bf Solution Sketch:}}
                      {\qed \end{trivlist}}
\newenvironment{code}{\begin{tabbing}
12345\=12345\=12345\=12345\=12345\=12345\=12345\=12345\= \kill }
{\end{tabbing}}

\newcommand{\eqref}[1]{Equation~(\ref{eq:#1})}

\newcommand{\hint}[1]{({\bf Hint}: {#1})}
%Put more macros here, as needed.
\newcommand{\room}{\medskip\ni}
\newcommand{\brak}[1]{\langle #1 \rangle}
\newcommand{\bit}{\{0,1\}}
\newcommand{\zo}{\{0,1\}}
\newcommand{\mod}{\mathsf{~mod~}}

\newcommand {\eps} {\varepsilon}

\newcommand{\nin}{\not\in}
\newcommand{\set}[1]{\{#1\}}
\renewcommand{\ni}{\noindent}
\renewcommand{\gets}{\leftarrow}
\renewcommand{\to}{\rightarrow}
\newcommand{\assign}{:=}

\newcommand{\AND}{\wedge}
\newcommand{\OR}{\vee}

\newcommand{\For}{\mbox{\bf For }}
\newcommand{\To}{\mbox{\bf to }}
\newcommand{\Do}{\mbox{\bf Do }}
\newcommand{\If}{\mbox{\bf If }}
\newcommand{\Then}{\mbox{\bf Then }}
\newcommand{\Else}{\mbox{\bf Else }}
\newcommand{\While}{\mbox{\bf While }}
\newcommand{\Repeat}{\mbox{\bf Repeat }}
\newcommand{\Until}{\mbox{\bf Until }}
\newcommand{\Return}{\mbox{\bf Return }}

\newcommand{\Enc}{\mathsf{Enc}}
\newcommand{\Dec}{\mathsf{Dec}}
\newcommand{\KeyGen}{\mathsf{KeyGen}}
\newcommand{\MAC}{\mathsf{MAC}}
\newcommand{\negl}{\mathsf{negl}}

\newcommand {\A} {\mathcal{A}}
\newcommand {\K} {\mathcal{K}}
\newcommand {\C} {\mathcal{C}}
\newcommand {\M} {\mathcal{M}}
\newcommand {\T} {\mathcal{T}}
\newcommand {\U} {\mathcal{U}}
\newcommand {\V} {\mathcal{V}}


\newcommand {\ZZ} {\mathbb{Z}}
\newcommand {\FF} {\mathbb{F}}
\newcommand {\NN} {\mathbb{N}}
\newcommand {\RR} {\mathbb{R}}

\newcommand {\AO}{\mathsf{AltOneSec}_A}
\newcommand {\OS}{\mathsf{OneSec}}
\newcommand{\Ta}{\mathsf{a}}
\newcommand{\Tb}{\mathsf{b}}
\newcommand {\DFA}{\texttt{DFA~}}

\def \SD {\mathsf{SD}}
\def \hinf {H_{\infty}}

\def \poly {\mathsf{poly}}

\begin{document}

\ifnum\me=0
\handout{PS\num}{\today}{Name: Ariel Hamlin}{Due: \due}{Solutions to Problem Set \num}

\fi
\ifnum\me=1
\handout{PS\num}{\today}{Name: Ariel Hamlin}{Due: \due}{Solution {\em Sketches} to Problem Set \num}
\fi
\ifnum\me=2
\handout{PS\num}{\today}{Lecturer:  Ariel Hamlin}{Due: \due}{Homework \num}
\fi

\textbf{Instructions:}
\begin{itemize}
\item The assignment has to be uploaded to Gradescope by the start of class. 
\item  \textit{You must write up your own solutions, in your own words.} 
\item You must turn in \textit{neat} work to Gradescope either written in a word processor such as Word, or typeset in LaTeX.
\end{itemize}

\newproblem{3 Passes}{20 pts}

For the paper ``Examining Teamwork: Evaluating Individual Contributions in
Collaborative Software Engineering Projects'', complete the three passes as we discussed in class. 

\begin{enumerate}
\item After the first pass answer the 5 `c' questions proposed in Keshev's paper.
\item After the second pass answer the following prompts:
\begin{enumerate}
\item Summarize the contributions and provide supporting evidence
\item Is the research problem significant? And is the contribution significant?
\item What are some of the strengths and weaknesses as presented by the authors of their contributions?
\item What are the open problems as identified by the authors?
\end{enumerate}
\item After a third pass, answer the following prompts:
\begin{enumerate}
\item What are one or two things you would want to do to follow up this research? This cannot be something the authors have already suggested
\item What are follow up reading would you want to do to better understand this area?

\end{enumerate}
\end{enumerate}


\end{document}



 
\end{document} 


